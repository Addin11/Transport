\section{Dynamika kwantowa w przestrzeni fazowej I- Konstrukcja kwantowej funkcji rozkładu w przestrzeni fazowej}
Oddziaływanie układu z otoczeniem skutkuje uciekaniem/przybywaniem energii z/do układu, np. przepuszczanie prądu przez układ skutkuje wydzieleniem ciepła; lub: ruch elektronów w sieci krystalicznej powoduje drgania tej sieci i wtedy gaz elektronowy i gaz fononowy zaczynają oddziaływać ze sobą.\\
Jak opisać taki problem? Rozwiązaniem jest statystyczne potraktowanie problemu.
\subsection{Statystyczne definicje w przestrzeni fazowej}
\subsubsection{operator gęstości} (inna nazwa: operator statystyczny) $\hat{\varrho}(t)$:
\begin{equation}\hat{\varrho}(t)=\sum_i \varrho_i \hat{P}_{\psi_i}(t)=
\sum_i \varrho_i|\psi_i(t)\rangle\langle\psi_i(t)|\end{equation}
przy czym:
\begin{equation}\sum_i\varrho_i=1\end{equation}
\subsubsection{Macierz gęstości} to operator statystyczny (operator gęstości) zapisany w odpowiedniej reprezentacji.\\
Niech $\alpha$ będzie zmienną ciągłą mającą własność:
\begin{itemize}
\item[1.] ortonormalności:
\begin{equation}\langle\alpha|\alpha'\rangle=\delta (\alpha-\alpha')\end{equation}
\item[2.] zupełności
\begin{equation}\int d\alpha|\alpha\rangle\langle\alpha|=\mathds{1}
\end{equation}
\end{itemize}
zatem stany $|\alpha\rangle$ tworzą bazę ortonormalną.\\
Wówczas macierz w reprezentacji $\alpha$ ma postać:
\begin{equation}\varrho(\alpha',\alpha,t)\equiv\langle\alpha'|\hat{\varrho}|\alpha\rangle=
\nonumber\end{equation}
\begin{equation}=\varrho(\alpha',\alpha,t)=
\langle\alpha'|{\scriptstyle{\sum_i}}|\psi_i(t)\rangle\varrho_i\langle\psi_i(t)|\alpha\rangle=\sum_i\varrho_i\langle\alpha'|\psi_i(t)\rangle\langle\psi_i(t)|\alpha\rangle=\nonumber\end{equation}
\begin{equation}
=\sum_i\varrho_i\psi_i(\alpha')\psi_i^*(\alpha)
\end{equation}
Podsumowując, w reprezentacji $\alpha$ macierz gęstości ma postać:
\begin{equation}\varrho(\alpha',\alpha,t)=\sum_i\varrho_i\psi_i(\alpha')\psi_i^*(\alpha)
\end{equation}
\subsection{Macierz gęstości w reprezentacji położeniowej i pędowej}
Niech $\alpha=\r$ lub $\alpha=\p$. Wówczas:
\begin{itemize}
\item macierz gęstości w reprezentacji położeniowej:
\begin{equation}\varrho(\r\,',r)=\sum_i\varrho_i\psi_i^*(\r)\psi_i(\r\,')
\end{equation}
\item macierz gęstości w reprezentacji pędowej:
\begin{equation}\varrho(\p\,',\p)=\sum_i\varrho_i\psi_i^*(\p)\psi_i(\p)\end{equation}
\end{itemize}
My jednak wiemy, że gęstość prawdopodobieństwa zależy od położenia $\r$ i pędu $\p$ jednocześnie. Jak zrealizować to "rozkroczenie" bazy? Najpierw nauczmy się przechodzić między obiema reprezentacjami.
\begin{itemize}
\item[1.] przejście $\r\rightarrow\p$ i  $\p\rightarrow\r$\\
Przejście $\r\rightarrow\p$ dla dowolnej funkcji $\psi$ to transformata Fouriera:
\begin{equation}\psi(\r,t)=\frac{1}{(2\pi\hbar)^3}\int d^3p\psi(\p,t)e^{\frac{i}{\hbar}\p\cdot\r}\end{equation}
a przejście $\p\rightarrow\r$ to odwrotna transformata Fouriera:
\begin{equation}\psi(\p,t)=\int d^3r\psi(\r,t)e^{-\frac{i}{\hbar}\p\cdot\r}
\end{equation}
\item[2.] przejście $\varrho(\p\,',\p,t)\rightarrow\varrho(\r\,',\r,t)$\\
Wykonajmy rachunek:
\begin{equation}\varrho(\p\,',\p,t)=\langle\p\,'|\hat{\varrho}(t)|\p\rangle\end{equation}
\begin{equation}\varrho(\p\,',\p,t)=\langle\p\,'|\mathds{1}\hat{\varrho}(t)\mathds{1}|\p\rangle
\end{equation}
\begin{equation}
\varrho(\p\,',\p,t)=\int d^3r'\int d^3r\langle\p\,'|\r\,'\rangle
\langle \r\,'|\hat{\varrho}(t)|\r\,'\rangle\langle\r|\p\rangle
\end{equation}
gdzie skorzystaliśmy z zupełności bazy: $\mathds{1}=|\r\rangle\langle\r|$. Dalej:
\begin{equation}
\varrho(\p\,',\p,t)=\int d^3r'\int d^3r\langle\p\,'|\r\,'\rangle
\varrho(\r\,',r,t)\langle\r|\p\rangle
\end{equation}
\begin{equation}
\varrho(\p\,',\p,t)=\int d^3r'\int d^3r\, e^{-\frac{i}{\hbar}\p\,'\cdot\r\,'}\langle\p\,'|\r\,'\rangle
\varrho(\r\,',r,t)\langle\r|\p\rangle e^{\frac{i}{\hbar}\p\cdot\r}
\end{equation}
\begin{equation}
\varrho(\p\,',\p,t)=\int d^3r'\int d^3r\,\langle\p\,'|\r\,'\rangle
\varrho(\r\,',r,t)\langle\r|\p\rangle e^{\frac{i}{\hbar}(\p\,'\cdot\r\,'-\r\cdot\p)}
\end{equation}
Wniosek: przejście  $\varrho(\p\,',\p,t)\rightarrow\varrho(\r\,',\r,t)$ to dwuwymiarowa transformata Fouriera.
\end{itemize}
\subsection{Interpretacja macierzy gęstości}
\subsubsection{elementy diagonalne}
\begin{equation}{\varrho}(\r,\r,t)=
\sum_i \varrho_i\psi_i^*(\r)\psi_i(\r)=\sum_i \varrho_i|\psi_i(\r)|^2=n(\r)\end{equation}
gdzie $n(\r)$ to rozkład gęstości elektronowej.\\
Wniosek: Elementy diagonalne macierzy gęstości w reprezentacji położeniowej dostarczają informacji o rozkładzie gęstości prawdopodobieństwa znalezienia cząstki w chwili t w położeniu $\r$ w poszczególnych stanach.\\
Analogicznie:
\begin{equation}{\varrho}(\p,\p,t)=n(\p)\end{equation}
gdzie $n(\p)$ to pędowy rozkład gęstości elektronowej.\\
Są to informacje \underline{klasyczne}.
\subsubsection{elementy pozadiagonalne}
Zgodnie z transformacją Fouriera:
\begin{equation}n(\r)={\varrho}(\r,\r,t)=\frac{1}{(2\pi\hbar)^3}\int d^3p\varrho(\p\,',p,t)e^{\frac{i}{\hbar}(\p\,'-\p)\cdot\r}
\end{equation}
zatem elementy pozadiagonalne macierzy gęstości w reprezentacji pędowej zawierają informacje o gęstości elektronowej $n(\r)$.\\
Analogicznie: 
\begin{equation}n(\p){\varrho}(\p,p,t)=\int d^3p\varrho(\r\,',\r,t)e^{-\frac{i}{\hbar}\p\cdot(\r\,'-\r)}
\end{equation}
elementy pozadiagonalne macierzy gęstości w reprezentacji położeniowej zawierają informacje o pędowej gęstości elektronowej $n(\p)$.\\
\textbf{Wniosek}: Elementy pozadiagonalne macierzy gęstości  dają informacje o korelacji położenia i pędu.\\
Zaczyna "pachnieć" zasadą nieokreśloności- ponieważ pod całkami występują iloczyny $\p$ i $\r$, to oznacza, że obu tych wielkości nie da się wyznaczyć jednocześnie. Tej informacji nie daje nam funkcja falowa.
\subsection{Macierz gęstości w reprezentacji $\alpha$}
Wartość oczekiwana zmiennej A jest równa śladowi macierzy:
\begin{equation}\langle A(t)\rangle=Tr\lbrace\hat{A}\hat{\varrho}(t)\rbrace
\end{equation}
Ponieważ $\alpha$ jest zmienną ciągłą, to obecną w definicji śladu macierzy sumę zamieniamy na całkę:
\begin{equation}\langle A(t)\rangle=\int d\alpha\langle\alpha|\hat{A}\hat{\varrho}(t)|\alpha\rangle\end{equation}
i kontynuujemy rachunek, wykorzystując zupełność bazy: $\mathds{1}=|\alpha'\rangle\langle\alpha'|$.
\begin{equation}\langle A(t)\rangle
=\int d\alpha\langle\alpha|\hat{A}\mathds{1}\hat{\varrho}(t)|\alpha\rangle
\end{equation}
\begin{equation}\langle A(t)\rangle=
\int d\alpha\int d\alpha'\langle\alpha|\hat{A}|\alpha'\rangle\langle\alpha'|\hat{\varrho}(t)|\alpha\rangle
\end{equation}
\begin{equation}\langle A(t)\rangle=
\int d\alpha\int d\alpha'A(\alpha,\alpha')\varrho(\alpha',\alpha,t)
\end{equation}
Zakładamy, że operator $\hat{A}$ jest lokalny:
\begin{equation}A(\alpha,\alpha'')=A(\alpha)\delta(\alpha-\alpha')
\end{equation}
Jest to tzw. przybliżenie lokalne. Wówczas:
\begin{equation}\langle A(t)\rangle=
\int d\alpha\int d\alpha'A(\alpha)\delta(\alpha-\alpha')\varrho(\alpha',\alpha,t)
\end{equation}
Z twierdzenia filtracyjnego dla delty Diraca:
\begin{equation}\langle A(t)\rangle=
\int d\alpha A(\alpha)\varrho(\alpha,\alpha,t)
\end{equation}
\begin{equation}\langle A(t)\rangle=
\int d\alpha A(\alpha)n(\alpha,t)
\end{equation}
gdzie $n(\alpha,t)$ to gęstość elektronowa w reprezentacji $\alpha$.\\
Zauważmy, że jeżeli $\hat{A}=\hat{mathds{1}}$, to:
\begin{equation}\langle 1\rangle=\int d\alpha \varrho(\alpha,\alpha,t)=
\int d^3\alpha\sum_i\varrho_i\psi^*(\alpha)\psi_i(\alpha)=\sum_i\varrho_i\int d^3\alpha\psi_i^*(\alpha)\psi_i(\alpha)=\sum_i\varrho_i\int d^3\alpha |\psi_i(\alpha)|^2=\sum_i \varrho_i
\end{equation}
zatem norma macierzy gęstości $\varrho$ jest indukowana normą funkcji falowej $\psi$.
\subsection{Podsumowanie- porównanie ujęcia klasycznego i kwantowego}
\begin{center}
  \begin{tabular}{ccc}  
    \toprule
        &  funkcja rozkładu & macierz gęstości \\
    \midrule
    warunek unormowania & $\int d^3r\int d^3pf\rpt=1$ &  $\int d^3r\varrho(\r,\r)=1$\\\\
  rozkład brzegowy & $n(\r,t)=\int d^3pf\rpt$ & $n(\r,t)=\sum_i\varrho_i\varrho(\r,\r,t)$\\\\
  				  & $n(\p,t)=\int d^3rf\rpt$ & $n(\p,t)=\sum_i\varrho_i\varrho(\p,\p,t)$\\\\
  wartość oczekiwana & $\langle A(t)\rangle=Tr\lbrace Af\rbrace=\int d^3r\int d^3pA(\r,\p)f\rpt$ & $\langle A(t)\rangle=Tr\lbrace \hat{A}\hat{\varrho}\rbrace=\int d\alpha A(\alpha)\varrho(\alpha,\alpha,t)$\\\\
  równanie ruchu &$ \partial_tf\rpt=\lbrace\hat{H}(\r,\p),f\rpt\rbrace$ &
 $ \frac{d}{dt}\hat{\varrho}(t)=\frac{1}{i\hbar}\Big[\hat{H},\hat{\varrho}(t)\Big]$\\\\
    \bottomrule
  \end{tabular}
\end{center}
\subsection{Wstęp do transformaty Wignera- wprowadzenie nowych zmiennych}
Przestrzeń Hilberta to zupełna przestrzeń liniowa ze zdefiniowanym iloczynem skalarnym.\\
Przestrzeń liniowa zawiera dodawanie wektorów, ale nie ich mnożenie (bo to wprowadzałoby działanie zewnętrzne- liczbę).\\
W przestrzeni Hilberta abstrakcyjnymi wektorami sią liczby zespolone, których mnożenie daje liczbę zespoloną- temu przestrzeń Hilberta zawiera zarówno dodawanie jak i mnożenie.\\
W 1932 roku Edward Wigner zastosował receptę Weyla w macierzy gęstości:
\begin{equation}\varrho(\r\,',\r)=\langle\r\,'|\hat{p}(t)|\r\rangle\end{equation}
Zakładając, że mamy stany czyste:
\begin{equation}\varrho(\r\,',\r)=
\langle\r\,'|\psi(t)\rangle\langle\psi(t)|\r\rangle\end{equation}
\begin{equation}\varrho(\r\,',\r,t)=\psi(\r\,',t)psi^*(\r,t)\end{equation}
Wigner zaproponował przejście do nowych zmiennych:
\begin{equation}(\r\,',\r)\rightarrow(\R,\S)\end{equation}
takich, że:
\begin{equation}\R=\frac{1}{2}(\r\,'+\r)\end{equation}
\begin{equation}\S=\r\,'-\r\end{equation}
Zauważmy, że $\R$ to środek ciężkości, a $\S$ to względne położenie.
Wigner wybrał akurat takie zmienne prawdopodobnie dlatego, że poszukiwał najprostszego rozwiązania.\\
W tych zmiennych:
\begin{equation}\r\,'=\R+\frac{1}{2}S\end{equation}
\begin{equation}\r=\R-\frac{1}{2}S\end{equation}
zatem:
\begin{equation}\varrho(\r\,',\r,t)=\varrho\Big(\R+\frac{1}{2}\S,\R-\frac{1}{2}\S,t\Big)\end{equation}
Zdefiniujmy funkcję pomocniczą:
\begin{equation}
W(\R,\vec{P})=\int d^3S\varrho\Big(\R+\frac{1}{2}\S,\R-\frac{1}{2}\S,t\Big)e^{-\frac{i}{\hbar}\vec{P}\cdot\S}
\end{equation}
jest to tzw. recepta Wayle'a.\\
W tej recepcie również leży pewna intuicja- zauważmy, że $\S$ jest związane z odległością $\r-\r\,'$, a zatem również z prędkością, więc i z pędem- dlatego mnożymy ją przez $\vec{P}$, czyli pęd.
