\section{Metody opisu klasycznej dynamiki cząstek}
W rozważaniach opuszczamy mechanikę Lagrangowską. 
\subsection{Mechanika Newtonowska}
\textbf{Siła Lorenza}
\begin{equation}\label{sila_lorenza}
	\vec{F}_l \arg = q [ \vec{E} \arg+\vec{v}(t)\times\vec{B}\arg].
\end{equation}
Jeżeli postać siły jest określona, to równanie ruchu możemy 
zapisać w postaci
\begin{equation}
	m\frac{d^2\vec{r}(t)}{dt^2}=\vec{F}_L \arg  .
\end{equation}
Zauważmy, że w mechanice Newtonowskiej nie ma ograniczenia na 
postać siły $\vec{F}_L$.

\textbf{Przykład - równanie Langevine'a}
$$ m\frac{d^2}{dt^2} \vec{r}(t) = \vec{F}_R - \gamma\vec{v}(t) +
\vec{\Gamma}(t),$$
gdzie $\vec{F}_R$ to siła regularna (np. od zewnętrznego pola 
elektrycznego, $\gamma$ to współczynnik tarcia, a $\vec{\Gamma(t)}$ 
to siła stochastyczna.

Rozwiązując równania Newtona otrzymujemy różne $\vec{r}(t)$. 
Oznaczmy 
\subsection{Mechanika Hamiltonowska}


