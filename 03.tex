\section{Metody opisu klasycznej dynamiki cząstek}
W rozważaniach opuszczamy mechanikę Lagrangowską. 
\subsection{Mechanika newtonowska}
\textbf{Siła Lorenza}
\begin{equation}\label{sila_lorenza}
	\vec{F}_l \arg = q [ \vec{E} \arg+\vec{v}(t)\times\vec{B}\arg].
\end{equation}
Jeżeli postać siły jest określona, to równanie ruchu możemy 
zapisać w postaci
\begin{equation}
	m\frac{d^2\vec{r}(t)}{dt^2}=\vec{F}_L \arg  .
\end{equation}
Zauważmy, że w mechanice Newtonowskiej nie ma ograniczenia na 
postać siły $\vec{F}_L$.
\textbf{Przykład - równanie Langevine'a}
$$ m\frac{d^2}{dt^2} \vec{r}(t) = \vec{F}_R - \gamma\vec{v}(t) +
\vec{\Gamma}(t),$$
gdzie $\vec{F}_R$ to siła regularna (np. od zewnętrznego pola 
elektrycznego, $\gamma$ to współczynnik tarcia, a $\vec{\Gamma(t)}$ 
to siła stochastyczna.
Rozwiązując równania Newtona otrzymujemy różne $\vec{r}(t)$. 
Oznaczmy przez $\{ \vec{r}(t) \}$ - zbiór rozwiązań równania Newtona
$\equiv$ PRZESTRZEŃ KONFIGURACYJNA.

RYSUNEK


$\left|\vec{r}(t)\right> $ - klasyczny stan cząstki w mechanice Newtona 
niewystarczający ze względu na brak determinizmu.

Stan cząstki opisany w spsób (trik dodający determinizm) $$\left| \vec{r}(t)
,\vec{v}(t) \right> \mbox{ - klasyczny stan cząstki}$$
$$
\begin{cases} 
	\dt \vec{r}(t)=\vec{v}(t) &\\
	m\dt \vec{v}(t)=\vec{F}(\vec{r},t)
\end{cases} + \mbox{war. początkowe (jednopunktowe)} 
\begin{cases}
	\vec{r}(t_0)=\vec{r}_0
	\vec{v}(t_0)=\vec{v}_0
\end{cases}
$$
\textbf{Uwaga}\\
Możemy określić $\vec{r}$ w chwili $t$, ale $\vec{v}$ okreśamy w otoczeniu $t$,
bo
$$\vec{v}_0 = \vec{v}(t_0) = \dt \vec{r}_{\Big|_{t=t_0}} = \lim_{\Delta t \to 0}
\frac{\vec{r}(t_0+\Delta t)-\vec{r}(t_0)}{\Delta t},$$
ewentualnie
$$\vec{v}_0 = \vec{v}(t_0) = \dt \vec{r}_{\Big|_{t=t_0}} = \lim_{\Delta t \to 0}
\frac{\vec{r}(t_0)-\vec{r}(t_0-\Delta t)}{\Delta t}.$$
\textbf{Wniosek}\\
Trikiem Tym uzyskujemy determinizm, z wyjątkiem infinitezymalnych zmian.

\subsection{Mechanika hamiltonowska}
W mechanice hamiltonowskiej nie używamy pojęcia siły, ale pojęcia potencjału,
co oznacza, że jest ona mniej ogólna.\\

\textbf{formalizm kanoniczny}\\
Funkcja Hamiltona: $ H\qpt$. Kosztem straty na ogólności, zyskujemy 
niezależność zmiennych uogólnionych $\vec{q}$ i $\vec{p}$.\\
$$ \left| \vec{q}(t), \vec{p}(t) \right> \mbox{ - klasyczny stan układu.} $$
Funkcja Hamiltona przybiera wartość całkowitej energii mechanicznej układu, jeżeli
siły sziałające na układ są potencjalne, a potencjał nie zależy od czasu.
$$ H\qp = \underbrace{J(\vec{q},\dot{\vec{q}}(\vec{q},\vec{p}))}_{\mbox{
część kinetyczna}} + \underbrace{U(\vec{q})}_{\mbox{część potencjalna}}.$$
